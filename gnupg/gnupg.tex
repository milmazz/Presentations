\documentclass{beamer}

\usetheme{Warsaw}
%\useoutertheme{}
%\useinnertheme{}
%\usefonttheme{}
%\usecolortheme{}

% Elimino los elementos de navegacion
\setbeamertemplate{navigation symbols}{}

\usepackage[utf8]{inputenc}
\usepackage[spanish]{babel}

% Redefinicion de teorema, definicion, ejemplo
\newtheorem{teorema}[theorem]{Teorema}
\newtheorem{definicion}[theorem]{Definición}
\newenvironment{ejemplo}{\begin{exampleblock}{Ejemplo}}{\end{exampleblock}}

\title{GNU Privacy Guard}
\subtitle{GnuPG}
\author[Milton Mazzarri]{Milton R. Mazzarri S. \\
\href{mailto:milmazz@gmail.com}{milmazz@gmail.com}
}
\date{14 de Octubre de 2010}

\hypersetup{colorlinks=true,linkcolor=blue}

\hypersetup{
pdfauthor={Milton R. Mazzarri S.},
pdftitle={GNU Privacy Guard}
}

\begin{document}

\frame{
\titlepage
}

\frame{
    \frametitle{¿Por qué?}

    \begin{itemize}[<+->]
        \item
        Si desea ser desarrollador oficial del Proyecto Debian, necesita una
        llave pública firmada por otros desarrolladores Debian.
        \item
        Promover políticas de seguridad en nuestras comunicaciones.
        \item
        Preparar al asistente a participar en las fiestas de firmado de llaves
        \end{itemize}
}

\frame{
    \frametitle{Fiestas de firmado de llaves}

    \begin{itemize}[<+->]
        \item
        Promueve la extensión de la red de confianza (\alert{WoT}).
        \item
        Oportunidad de discutir políticas alrededor de la criptografia fuerte.
        \item
        Discusiones acerca de implementación de tecnologías de cifrado.
    \end{itemize}
}

\frame{
    \frametitle{Criptografía}

    \begin{itemize}[<+->]
        \item
        Disciplina de las matemáticas que ocupan la seguridad de la
        información.
        \item
        Cifrado.
        \item
        Autenticación.
        \item
        Control de accesos.
        \item
        Ocultar el significado del mensaje y no su existencia.
    \end{itemize}
}

\frame{
    \frametitle{Claves simétricas}

    \begin{itemize}[<+->]
        \item
        Una clave para cifrar el contenido.
        \item
        Debe utilizar la misma clave para descifrar.
        \item
        Como transfiere de manera segura esa contraseña.
    \end{itemize}
}

\frame{
    \frametitle{Claves asimétricas}

    \begin{itemize}[<+->]
        \item
        Clave pública.
        \item
        Clave privada.
        \item
        El remitente utiliza la clave pública del destinatario para cifrar el
        mensaje.
    \end{itemize}
}

\frame{
    \frametitle{Firmas digitales}

    \begin{itemize}[<+->]
        \item
        Permite verificar la autoría del mensaje.
        \item
        El destinatario comprueba que el ``supuesto'' remitente es quien afirma
        ser.
        \item
        El remitente firma el mensaje con su llave privada.
    \end{itemize}
}

\frame{
    \frametitle{Beneficios}

    \begin{itemize}[<+->]
        \item
        Evitar riesgos de seguridad.
        \item
        Verificar el origen de los datos.
        \item
        Es muy sencillo falsificar la dirección y nombre del remitente.
        \item
        Es imposible falsificar la firma digital del remitente.
        \item
        Verificar la integridad del mensaje.
        \item
        Cualquier manipulación del contenido arrojará un error en la
        verificación.
    \end{itemize}
}

\frame{
    \frametitle{Anillos de confianza}

    \begin{itemize}[<+->]
        \item
        Para confiar en una clave debemos verificar quien le ha firmado.
        \item
        Cuando se confie plenamente en la clave del firmante, se puede confiar
        en la firma que acompaña a la clave de un tercero.
        \item
        Tener certeza que la clave es la correcta por medio del fingerprint o
        huella digital.
    \end{itemize}
}

\frame{
    \frametitle{¿Es GnuPG la panacea?}

    \begin{itemize}[<+->]
        \item
        Es un paso hacia un sistema más seguro.
        \item
        No basta con la instalación de un programa criptográfico.
        \item
        Debe pensar en la seguridad en general.
    \end{itemize}
}

\begin{frame}[fragile]
    \frametitle{Instalación de GnuPG}

    \begin{ejemplo}
    \begin{verbatim}
    # aptitude install gnupg
    \end{verbatim}
    \end{ejemplo}
\end{frame}

\begin{frame}[fragile]
    \frametitle{Uso y gestión de Llaves (I)}

    \begin{ejemplo}
    \begin{verbatim}
    $ gpg --gen-key
    \end{verbatim}
    \end{ejemplo}

    \begin{itemize}[<+->]
        \item
        Se genera un par de claves (privada y pública).
        \item
        El algoritmo recomendado por \alert{GnuPG} es \texttt{DSA/ElGamal} ya
        que no está patentado.
        \item
        Longitud de clave (1024, 4096, 2048).
        \item
        Nombre y Apellidos, entre parentesis el apodo.
    \end{itemize}
\end{frame}

\frame{
    \frametitle{Uso y gestion de Llaves (II)}

    \begin{itemize}[<+->]
        \item
        El último paso consiste en el ingreso de \alert{una frase de paso} o
        \emph{passphare}.
        \item
        Su frase debe ser:
        \begin{itemize}
            \item
            Larga
            \item
            Combinar mayúsculas, minúsculas, números.
            \item
            Trate de utilizar la letra ñ.
            \item
            Dificil de adivinar.
        \end{itemize}
    \end{itemize}
}

\frame{
    \frametitle{Certificados de revocación}

    \begin{itemize}[<+->]
        \item
        Después de crear sus llaves, se recomienda crear certificados de
        revocación.
        \item
        Es necesario en caso de olvidar la frase de paso.
        \item
        Oportuno usar cuando la llave privada está en riesgo.
    \end{itemize}
}

\begin{frame}[fragile]
    \frametitle{Exportar llaves (I)}

    \begin{ejemplo}
    \begin{semiverbatim}
    \uncover<1->{$ gpg --export [UID]}
    \uncover<2->{$ gpg --keyserver pgp.mit.edu --send-keys UID}
    \end{semiverbatim}
    \end{ejemplo}

    \begin{itemize}
        \item<1->
        Si no especifica un UID se exportaran todas las claves presentes en el
        anillo de llaves.
        \item<2->
        Se puede exportar a un servidor de claves públicas.
    \end{itemize}
\end{frame} 

\begin{frame}[fragile]
    \frametitle{Exportar llaves (II)}

    \begin{ejemplo}
    \begin{verbatim}
    $ gpg --export-secret-keys UID > clave.asc
    \end{verbatim}
    \end{ejemplo}

    \begin{itemize}[<+->]
        \item
        Exporta la llave privada a un medio fisico
        \item
        Almacene \texttt{clave.asc} en un lugar seguro.
    \end{itemize}
\end{frame}

\begin{frame}[fragile]
    \frametitle{Crear certificado de revocación}

    \begin{ejemplo}
    \begin{verbatim}
    $ gpg --gen-revoke UID
    \end{verbatim}
    \end{ejemplo}

    Ahora Deberá:

    \begin{itemize}[<+->]
        \item
        Imprimirlo.
        \item
        Exportarlo a un medio fisico.
    \end{itemize}
\end{frame}

\begin{frame}[fragile]
    \frametitle{Importar claves}

    \begin{ejemplo}
    \begin{verbatim}
    $ gpg --import [fichero]
    \end{verbatim}
    \end{ejemplo}

    Cuando se recibe la clave pública de otra persona hay que añadirla a la
    base de datos (anillo) para lograr hacer uso de ella.
\end{frame}

\frame{
    \frametitle{Administrar las claves (I)}

    \begin{description}
        \item[\texttt{gpg --list-keys}] muestra las claves presentes.
        \item[\texttt{gpg --fingerprint}] muestra las huellas digitales.
        \item[\texttt{gpg --delete-key UID}] elimina un \texttt{UID} dado.
    \end{description}
}

\begin{frame}[fragile]
    \frametitle{Administrar las claves (II)}

    \begin{ejemplo}
    \begin{verbatim}
    $ gpg --edit-key UID
    \end{verbatim}
    \end{ejemplo}

    \begin{itemize}
        \item
        Editar la fecha de caducidad.
        \item
        Añadir una huella digital.
        \item
        Firmar la clave.
    \end{itemize}
\end{frame}

\frame{
    \frametitle{Frontends}

    \begin{itemize}
        \item
        \url{http://enigmail.mozdev.org/}
        \item
        \url{http://www.gnupg.org/(es)/related_software/frontends.html}
        \item
        \url{http://people.redhat.com/~walters/gnome-gpg}
        \item
        \url{http://developer.kde.org/~kgpg}
    \end{itemize}
}

\frame{
    \frametitle{¡Muchas gracias!}

    ¿Desea contactarme?

    \begin{description}
    \item[correo] \href{mailto:milmazz@gmail.com}{milmazz@gmail.com}
    \item[blog] \href{http://blog.milmazz.com.ve}{http://blog.milmazz.com.ve}
    \item[nombre] Milton R. Mazzarri S.
    \end{description}
}
\end{document}
