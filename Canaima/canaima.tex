% Copyright (c) 2010 Milton Mazzarri <milmazz@gmail.com>
% All rights reserved.
% 
% Redistribution and use in source and binary forms, with or without
% modification, are permitted provided that the following conditions
% are met:
% 1. Redistributions of source code must retain the above copyright
%    notice, this list of conditions and the following disclaimer.
% 2. Redistributions in binary form must reproduce the above copyright
%    notice, this list of conditions and the following disclaimer in the
%    documentation and/or other materials provided with the distribution.
% 3. The name of the author may not be used to endorse or promote products
%    derived from this software without specific prior written permission.
% 
% THIS SOFTWARE IS PROVIDED BY THE AUTHOR ``AS IS'' AND ANY EXPRESS OR
% IMPLIED WARRANTIES, INCLUDING, BUT NOT LIMITED TO, THE IMPLIED WARRANTIES
% OF MERCHANTABILITY AND FITNESS FOR A PARTICULAR PURPOSE ARE DISCLAIMED.
% IN NO EVENT SHALL THE AUTHOR BE LIABLE FOR ANY DIRECT, INDIRECT,
% INCIDENTAL, SPECIAL, EXEMPLARY, OR CONSEQUENTIAL DAMAGES (INCLUDING, BUT
% NOT LIMITED TO, PROCUREMENT OF SUBSTITUTE GOODS OR SERVICES; LOSS OF USE,
% DATA, OR PROFITS; OR BUSINESS INTERRUPTION) HOWEVER CAUSED AND ON ANY
% THEORY OF LIABILITY, WHETHER IN CONTRACT, STRICT LIABILITY, OR TORT
% (INCLUDING NEGLIGENCE OR OTHERWISE) ARISING IN ANY WAY OUT OF THE USE OF
% THIS SOFTWARE, EVEN IF ADVISED OF THE POSSIBILITY OF SUCH DAMAGE.

\documentclass{beamer}

\mode<presentation>
{
  \usetheme{Frankfurt}
  \setbeamercovered{transparent}
  \useoutertheme{tree}
  \useinnertheme{rounded}
}

\usepackage[spanish]{babel}
\usepackage[utf8]{inputenc}

\usepackage{times}
\usepackage[T1]{fontenc}

% Redefinicion de teorema, definicion, ejemplo
\newtheorem{teorema}{Teorema}
\newtheorem{definicion}{Definición}
\newtheorem{ejemplo}{Ejemplo}

\hypersetup{colorlinks=true,linkcolor=red}

\title{Canaima GNU/Linux}
\subtitle{Introducción}
\author[Milton Mazzarri]{Milton Mazzarri \\
\texttt{<milmazz@gmail.com>}}
\institute[CUHELAV]{Colegio Universitario Hotel Escuela de los Andes
Venezolanos}
\date{Julio, 2010}

\subject{Canaima GNU/Linux}

 \AtBeginSection[]
{
  \begin{frame}<beamer>[allowframebreaks]{Contenido}
    \tableofcontents[currentsection,currentsubsection]
  \end{frame}
}

% If you wish to uncover everything in a step-wise fashion, uncomment
% the following command: 
% \beamerdefaultoverlayspecification{<+->}

\begin{document}

\begin{frame}
    \titlepage
\end{frame}

% \begin{frame}[allowframebreaks]{Contenido}
%    \tableofcontents[pausesections]
% \end{frame}

\section{Software Libre}

\frame{
    \frametitle{Software Libre}

    \begin{description}
        \item[Libertad 0] La libertad de ejecutar el programa, para cualquier
        propósito.
        \item[Libertad 1] La libertad de estudiar cómo trabaja el programa, y
        cambiarlo para que haga lo que usted quiera\footnote{ El acceso al
        código fuente es una condición necesaria para ello.}. 
        \item[Libertad 2] La libertad de redistribuir copias para que pueda
        ayudar al prójimo. 
        \item[Libertad 3] La libertad de distribuir copias de sus versiones
        modificadas a terceros\footnote{Si lo hace, puede dar a toda la
        comunidad una oportunidad de beneficiarse de sus cambios}. 
    \end{description}
}

\section{Definición}

\frame{
    \frametitle{Canaima GNU/Linux en su fase inicial}

    \begin{definicion}
    Surge inicialmente como respuesta a las necesidades ofimáticas y de otros
    procesos productivos de los usuarios finales de la \emph{Administración Pública
    Nacional} (APN), y para dar cumplimiento al decreto presidencial Nro. 3.390
    sobre el uso de Tecnologías de Información Libres en la APN.
    \end{definicion}
}

\frame{
    \frametitle{¿Qué es Canaima GNU/Linux hoy día?}

    \begin{definicion}
    Proyecto socio-tecnológico abierto, construido de forma colectiva, centrado
    en el desarrollo de herramientas y modelos productivos basados en las
    Tecnologías de Información Libres (TIL) de software y sistemas operativos
    cuyo objetivo es generar capacidades nacionales, desarrollo endógeno,
    apropiación y promoción del libre conocimiento, sin perder su motivo
    original: la construcción de una Nación venezolana tecnológicamente
    preparada.
    \end{definicion}
}

\section{Características}

\frame{
    \frametitle{Características principales}

    \begin{itemize}
        \item Totalmente desarrollada en Software Libre.
        \item No está limitada al uso en la APN, puede ser usado por cualquier
        persona.
        \item Se encuentra equipado con herramientas ofimáticas:
        \begin{itemize}
            \item Procesador de palabras
            \item Hojas de cálculo
            \item Presentaciones
        \end{itemize}
        \item También incluye herramientas para:
        \begin{itemize}
            \item Diseño gráfico
            \item Planificación de proyectos
            \item Bases de datos
        \end{itemize}
    \end{itemize}
}

\frame{
    \frametitle{Características principales}

    \begin{itemize}
        \item Facilita la interacción con Internet a través de su navegador
        Web, gestor de correo electrónico y aplicaciones para realizar llamadas
        telefónicas por la red.
        \item Es segura, Canaima está basada en la versión estable de Debian
        GNU/Linux, la cual pasa por una serie de procesos y pruebas rigurosas
        de calidad.
        \item Realizada en Venezuela por talento nacional.
    \end{itemize}
}

\frame{
    \frametitle{Distribución basada en Debian GNU}

    \begin{itemize}
        \item Arquitectura soportadas: x86, PowerPC, IA-32, S/390, HP PA-RISC,
        SPARC, IA-64, m68k, MIPS, MIPSEL, AMD64, EMT64T, Alpha.
        \item Ofrece soporte más allá del kernel \alert{linux}: GNU/Hurd,
        GNU/NetBSD, GNU/kFreeBSD. 
    \end{itemize}
}

\frame{
    \frametitle{Measuring Lenny, The size of Debian 5.0}

    Investigación llevada a cabo por la Universidad Rey Juan Carlos, Madrid,
    España. \url{http://libresoft.es/debian-counting/}

    \begin{itemize}
        \item COnstructive COst Model
        \begin{itemize}
            \item Técnica de estimación usada en
            Ingeniería clásica del Software.
            \item Medir líneas de código.
            \item Medir salarios al mes.
        \end{itemize}
    \end{itemize}
}

\frame{
    \frametitle{¿Cuanto costaría escribir Debian GNU/Linux 5.0?}

    Asuma que está escribiendo este Sistema Operativo desde cero, no la unión
    de proyectos independientes.

    \begin{itemize}
        \item Esfuerzo estimado: 122.030,91 personas/año.
        % NOTA: Person-Months = 2.4 * (KSLOC**1.05)
        % DEBIAN 4.0: \item Esfuerzo estimado: 73400 personas/año
        \item Tiempo estimado: 45,89 años
        % NOTA: Months = 2.5 * (person-months**0.38)
        % DEBIAN 4.0 \item Tiempo estimado: cerca de 9 años
        \item Costos estimados: 315.195.900.000 USD
        % NOTA: average salary = $56,286/year, overhead = 2.40
        % DEBIAN 4.0: \item Costos estimados: 6.7 billones de USD
        \item Otros aspectos:
        \begin{itemize}
            \item 323.551.126 SLOC
            % DEBIAN 4.0: \item 283.000.000 SLOC
            \item ANSI C, C++, Shell scripts, Java, Python, Perl y LISP son
            los lenguajes más usados.
            % DEBIAN 4.0: \item C, C++, Shell scripts, Java, Perl y LISP son
            % los lenguajes más usados
            \item Debian 5.0 representa la compilación más grandes de FLOSS
            % \item Debian 4.0 representa la compilación más grandes de FLOSS
        \end{itemize}
    \end{itemize}
}

\frame{
    \frametitle{¿Por qué Canaima GNU/Linux?}

    \begin{itemize}
        \item Apalancar el proceso de adopción de Tecnologías Libres en el
        país.
        \item Fomentar una estructura organizada de desarrollo nacional.
        \item Ofrecer soporte técnico de calidad con la ayuda de la comunidad
        de Software Libre del país.
        \item Ciclo de desarrollo propio.
    \end{itemize}
}

\frame{
    \frametitle{Sabores de Canaima GNU/Linux}

    \begin{description}
        \item[Canaima Educativo] Proveer recursos de aprendizaje que buscan
        impulsar la interacción entre el niño y el
        computador\footnote{\href{http://wiki.canaima.softwarelibre.gob.ve/wiki/index.php/Canaima_Educativo}{Más
        información sobre Canaima Educativo}}.
        \item[Canaima Accesible] Iniciativa que persigue cubrir
        los requerimientos de personas con algún tipo de
        discapacidad\footnote{\href{http://wiki.canaima.softwarelibre.gob.ve/wiki/index.php/Canaima_Accesible}{Más
        información sobre Canaima Accesible}}.
        \end{description}
}

\frame{
    \frametitle{Sabores de Canaima GNU/Linux}

    \begin{description}
        \item[Canaima Colibrí] Trabajar en computadoras de bajos recursos sin
        afectar la facilidad y eficiencia en su uso, es un CD ``en vivo'' lo que
        permite probar la distribución sin necesidad de
        instalarla\footnote{\href{http://forja.softwarelibre.gob.ve/projects/colibri/}{Más
        información sobre Canaima Colibrí}}.
        \item[Canaima Forense] Dedicada al área de la informática forense,
        orientada a investigadores
        policiales\footnote{\href{http://forja.softwarelibre.gob.ve/projects/canaimaforense/}{Más
        información sobre Canaima Forense}}.
    \end{description}
}

\frame{
    \frametitle{Último lanzamiento de Canaima GNU/Linux: 2.1}

    \begin{itemize}
        \item Incorporación del paradigma de medios ``vivo'' instalables,
        tecnología que permite la ejecución de Canaima sin necesidad de alterar
        el sistema existente. Esta manera de disfrutar Canaima GNU/Linux, se
        suma a la forma tradicional de instalar permanentemente la distribución
        en el computador
        \item Facilidad provista a usuarios de sistemas de operación privativos
        (tales como Microsoft Windows), la cual ofrece una forma transparente
        de migración hacia el sistema libre Canaima
        \end{itemize}
}

\frame{
    \frametitle{Último lanzamiento de Canaima GNU/Linux: 2.1}

    \begin{itemize}
        \item Interfaz gráfica de usuario sencilla para una experiencia
        mejorada en la partición de disco del computador
        \item Mejora del sistema de chequeo de disco en el arranque, cuidando
        la impresión del usuario al momento de esa actividad
        \item Actualización general de programas informáticos provenientes de
        la distribución matriz Debian GNU/Linux
        \item Soporte a las arquitecturas de hardware 32 y 64 \emph{bits}
    \end{itemize}
}

\section{Colaborar}

\frame{
    \frametitle{Cayapa Comunitaria}

    \begin{definicion}
    Término utilizado por los campesinos para referirse al trabajo que se hace
    entre muchos sin ánimos de lucro, mayormente para acometer una tarea de la
    comunidad.
    \end{definicion}
}

\frame{
    \frametitle{Cayapa Canaima}

    \begin{definicion}
    Es un evento socio-tecnológico, enfocado a tratar una variedad de temas
    relacionados con el desarrollo de nuevas aplicaciones para la
    Distribución, corrección de errores, pruebas de nuevos desarrollos de
    programas informáticos, así como también la discusión de aspectos sociales
    y tecnológicos involucrados alrededor de la Comunidad de Software Libre.
    \end{definicion}
}

\frame{
    \frametitle{Cayapas Canaima}

    \begin{enumerate}
        \item Punto Fijo -- Diciembre 2008
        \item Mérida -- Agosto 2009
        \item Barquisimeto -- Abril 2010
        \item \alert{San Juan de los Morros -- Octubre 2010}
    \end{enumerate}
}

\frame{
    \frametitle{¿Necesito ser programador para colaborar en Canaima?}

    El modelo de trabajo del Proyecto Canaima maneja siete (7) roles básicos:

    \begin{itemize}
        \item Desarrollador
        \item Relaciones públicas
        \item Documentador
        \item Administrador de sistemas
        \item Facilitador
        \item Diseñador
        \item Traductor
    \end{itemize}
}

\section{Enlaces}

\frame{
    \frametitle{Enlaces de interés}

    \begin{itemize}
        \item
        \href{http://canaima.softwarelibre.gob.ve:8080/canaima_cms/soporte/descargas/}{Descargas}
        \item \href{http://canaima.softwarelibre.gob.ve/cms/soporte}{Soporte}
        \begin{itemize}
            \item Enciclopedia comunitaria
            \item Sala de Chat -- IRC
            \item Listas de correo
            \item Reporte de errores
        \end{itemize}
        \item
        \href{http://canaima.softwarelibre.gob.ve/cms/unirse-al-proyecto}{Unirse
        al proyecto}
        \item
        \href{http://canaima.softwarelibre.gob.ve:8080/canaima_cms/documentos/FAQ.pdf/at_download/file}{Preguntas
        de Uso Frecuente}
    \end{itemize}
}


\frame{
    \frametitle{¡Muchas Gracias!}

    \begin{description}
        \item[Nombre] Milton Mazzarri
        \item[Correo] \href{mailto:milmazz@gmail.com}{milmazz@gmail.com}
        \item[Blog] \url{http://blog.milmazz.com.ve}
        \item[Twitter] \url{http://twitter.com/milmazz}
        \item[Presentaciones] \url{http://www.scribd.com/milmazz}
    \end{description}
}
\end{document}
